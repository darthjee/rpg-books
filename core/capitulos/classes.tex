\chapter{Classes}
\begin{multicols}{2}
Para se viver em um mundo cheio de perigos � nescess�rio que
um her�is tenha as habilidades nescess�ria que podem
ser a diferen�a entre a vida e a morte.

A classe selecionada no inicio da jornada representa sua profiss�o,
e determinar� os poss�veis caminhos para este aventureiro, sejam,
estes, os conhecimentos arcanos de Mago, as habilidades de combate de um
Guerreiro ou os conhecimento selvagens de um Druida.
\section{As Classes}
As classes b�sicas dispon�veis para todos os her�is s�o as
classes a seguir:
\begin{description}
 \item[Guerreiro:] Mestres do combate armado, vers�teis
 e com incrivel habilidade de expecializa��o.
 \item[Mago:] Estudiosos das for�as oultas que movem o universo.
 \item[Especialista:] O especialista prefere n\~ao utilizar-se
 do combate direto, preferindo a estrat\'egia e apoio t\'atico
 e de suprimentos aos seus companheiros.
\end{description}
\subsection{Personagens Multiclasse}
Conforme seu personagem avan�a na carreira de her�i, ele
pode adicionar novas classes. Adicionanado novas classes
Torna o personagem mais vers�til combinando assim seus pontos
fortes, entretanto, isso tambem limita as habilidades individuais
de cada classe.

Um mago, por exemplo, pode decidir abra�ar a carreira de
guerreiro para melhorar suas habilidades de combate
e sua resist�ncia f�sica, ou pode decidir abra�ar
as habilidades de de um ladino para se tornar uma m�quina
de habiliddades vers�teis, entretanto, ao se fazer tal
escolha, o personagem sacrifica a evolu��o de seus poderes
arcanos que teria se se mantivesse "puro".

\section{Guerreiro}

O guerreiro representa todos aqueles que treinam
primando a excel�ncia do combate f�sico, aqueles que
veem na

\section{Especialista}
\begin{description}
\item[Adventures]
\item[Characteristics]
\item[Aligment]
\item[Religion]
\item[Background]
\item[Races]
\item[Other Classes]
\item[Role]
\end{description}

\subsection{Informa\c{c}\~oes de jogo}
\begin{description}
\item[Habilidades:]
Inteligencia \'e uma habilidade chave para que o
especialista possa ter acesso a todas as suas habilidades.

Durante o combate, o especialista faz uso de sua intelig\^encia
e dextreza, se mantendo longe do combate direto onde pode
encontrar-se em desvantagem procurando sempre manipular as
condi\c{c}\~oes de combate a seu favor e a favor de seu grupo
\item[Alinhamento:]
\item[Dados de Vida:] d6
\subsubsection{Per\'icias de classes}
As per\'icias de classe (e a habilidade chave para cada per\'icia) s\~ao
Abrir Fechaduras (Des),
Avalia\c{c}\~ao (Int),
  Concentra\c{c}\~ao (Con),
  Conhecimento (Arquitetura)(Int),
  Conhecimento (Natureza)(Int),
  Decifrar escrita (Int),
  Nata\c{c}\~ao (For),
  Observar (Sab),
  Obter informa\c{c}\~oes (Car)
  Of\'icios (Int),
  Operar mecanismo (Int),
  Ouvir (Sab),
  Procurar (Int),
  Profiss\~ao (Sab)
  Usar cordas (Des)
  Usar instrumentos magicos (Car),
\end{description}

\subsubsection{Caracter\'isticas da classe}
\begin{description}
\item[Armas e Armaduras]
Os especialistas s\~ao treinados com todas as armas simples alem
de espada curta,
arco curto, martelo leve e machado de m\~ao leve.
Especialistas sabem utilizar armaduras leves mas n\~ao escudos.

\item[Area de especialidade]
O especialista se concentra em suas \'areas de estudo
obtendo assim um sucesso extra com assuntos relacionados
\`a sua area.

O especialista escolhe no primeiro n\'ivel (e sempre que escolher
    uma \'area extra de especialidade) uma dentre
as \'areas de estudo recebendo um b\^onus de +2 nos testes
de conhecimento, of\'icios, profiss\~ao e avalia\c{c}\~ao
nos testes relacionados \`a sua area de especialidade.

Dentre as \'areas poss\'iveis est\~ao e seus b\^onus est\~ao:

\begin{description}
\item[Engenharia]
O especialista conhece muito de constru\c{c}\~oes e mec\^anica, capaz
de construir desde catapultas \`a castelos.

O engenheiro sabe utilizar as bestas de repeti\c{c}\~ao leve
e pesadas.

\item[Armas e Armaduras]
O armeiro conhece muito de combate e seus equipamentos, estando sempre
envolvido na constru\c{c}\~ao e aprimoramento dos equipamentos.

O armeiro sabe utilizar armaduras m\'edias e escudos leves.

O armeiro tabem tem cavalgar como uma habilidade de classe.

\item[Qu\'imica]
O qu\'imico entende muito dos elementos brutos encontrados
na natureza e como criar efeitos potentes com estes.

O qu\'imico esta imune contra envenenamentos durante
a produ\c{c}\~ao de seus compostos

O qu\'imico ganha o talento produzir itens alquimicos.

\item[Herbalismo]
O Herbalista esta em contato com a natureza aprendendo sobre
as propriedades das plantas e outras subst\^ancias.

O herbalista recebe como pericia de classe Cura e Sobreviv\^encia.

\item[Artes]
O artista \'e de longe um aventureiro, estando muito mais ligado
a sua produ\c{c}\~ao artistica do que a mmasmorras

O artista recebe as per\'icias Falsifica\c{c}\~ao e 1 tipo de
performance como per\'icias de classe.

\item[Metalurgia]
O metalurgico conhece tudo sobre metais, suas combina\c{c}\~oes, forja, etc...

\item[Artificio]
O artifice sabe como unir mmagia e itens comuns.

\item[Tailoring]
O Tailor sabe como manipular tecidos, couro, etc..

\item[Gastronomia]
O especialista sabe e entende tudo sobre comida.
\end{description}

\item[Conhecimento de especialista] O especialista pode fazer um teste
para reconhecer um objeto de sua \'area de especialidade
(qu\'imica ou engenharia escolhida no preimeiro n\'ivel).

O especialista joga seu n\'ivel de especialista + seu bonus de
intelig\^encia contra uma DC dependendo de qu\~ao comum \'o o
objeto (utilize a tabela de conhecimento de bardo)

O especialista deve analisar o objeto durante um tempo minimo de 1 minuto
para determinar o que sabe sobre ele.

O especialista n\~ao pode escolher tirar 10 ou 20 neste teste.

\item[Reserva de of\'icio]
O especialista tem pontos de reserva para serem utilizadas na sua produ\c{c}\~ao.
Os pontos de reserva podem ser utilizados como pontos de craft descritos no livro
Unearthed Arcana pg 97.

O Especialista recebe, a cada n\'ivel 100xn\'ivel de especialista pontos de reserva.

Um especialista que comece em um n\'ivel acima do primeiro, tem como reserva
100xo n\'ivel de especialista mais quaisquer pontos extras dados por
talentos adquiridos.

\item[Encontrar armadilhas]
A partir do segundo n\'ivel, o especialista pode
utilizar a per\'icia Procurar para encontrar
armadilhas cuja a Classe de dificuldade seja superior a 20.
Encopntrar armadilhas mundanas tera CD 20 ou superior quando esta
esta bem escondida. Encontrar armadilhas m\'agicas ter\'a uma
dificuldade de 25+n\'ivel da magia utilizada na sua cria\c{c}\~ao.

O especialista pode utilizar a per\'icia Operar mecanismos para
desarmar armadilhas m\'agicas.
uma armadilha m\'agica ter\'a CD 25 + o n\'ivel da magia utilizada
na sua cria\c{c}\~ao.

Um especialista que obtenha um sucesso com uma margem igual ou superior
a 10 para desativar uma armadilha conseguira estuda-la, entender
seu funcionamento e super\'a-la (junto com seus companheiros) sem desarm\'a-la

O especialista pode utilizar esta habilidade somente com armadilhas
que utilizem como componente principal, elementos de sua area de especialidade

\item[Combate preciso]
Iniciando no segundo n\'ivel, o especialista utiliza sua estrat\'egia de
combate para golpear precisamente.

Utilizando sua a\c{c}\~ao de movimento para mirar, o especialista adciona
metade de seu n\'ivel (at\'e o limite de seu b\^onus de intelig\^encia)
na sua pr\'oxima jogada de ataque.

O especialista perde este b\^onus caso seja atigido por um atque e
falhe em um teste de concentra\c{c}\~ao de CD igual ao dano recebido.

\item[Ataque Preciso Poderoso]
Sempre que o especialista utilizar o combate preciso, o especialista
pode transfirir pontos de sua jogada de ataque para o dano.

O especialista n\~ao pode transferis, desta forma, mais pontos
de sua jogada de ataque do que ele tem de b\^onus de combate preciso.

O especialista pode realizar um ataque preciso poderoso com armas
de ataque a dist\^ancia somente quando o alvo estiver a uma dist\^ancia
de 9m ou menos.

\item[Caracter\'istica b\^onus] a partir do \(5^o\) n\'ivel e a cada
3 n\'iveis subsequentes, o especialista ganha uma caracter\'istica b\^onus.
As caracter\'isticas b\^onus dispon\'iveis s\~ao:

\begin{description}
\item[Area de espcialidade] o especialista pode escolher uma \'area de especialidade
adcional.

Esta caracter\'istica pode ser escolhida multiplas vezes, cada vez representando uma
\'area de conhecimento nova.

\item[Fabricar armas obra prima]
\item[Fabricar armas de long alcance obra prima]
\item[Fabricar armaduras obra prima]
\item[Maestria de per\'icia] O especialista se torna t\~ao preciso
no uso de suas habilidades, que ele pode contar com elas
mesmo sob condi\c{c}\~oes extremas.

Ao selecionar esta caracter\'istica, o especialista pode selecionar
um n\'umero de per\'icias igual a 3 + o seu modificador de
intelig\^encia. Quando fizer um teste com qualquer umas destas
per\'icias, ele pode escolher 10 na jogada mesmo em uma
sita\c{c}\~ao que ele normalmenten\~ao poderia.

Esta caracter\'istica pode ser escolhida multiplas vezes, selecionando
per\'icias adcionas cada vez que esta caracteristica \'e selecionada
\end{description}

\item[Ataque preciso poderoso aprimorado]
O especialista pode utilizar seu ataque preciso poderoso mesmo quando
n\~ao conseguir utilizar o ataque preciso.

\item[Combate preciso aprimorado]
Mirar para utilizar o combate aprimorado passa a ser uma
a\c{c}\~ao r\'apida ao inv\'es de uma a\c{c}\~ao de movimento

\end{description}

\end{multicols}
\begin{table}[htb]
\begin{center}
\begin{tabular}{|c|c|c|c|c|c|}
\hline
Level & ABB & Fort & Ref & Will & Especial \\
\hline
\(1^o\) & +0 & +2 & +0 & +0 & Area de especialidade \\
 & & & & & Conhecimento de especialista \\
 & & & & & Reserva de of\'icio \\
\hline
\(2^o\) & +1 & +3 & &  & Encontrar armadilhas \\
    & & & & & Combate preciso +1 \\
\hline
\(3^o\) & +2 & +3 & &  & Ataque preciso poderoso \\
\hline
\(4^o\) & +3 & +4 & &  & Combate preciso +2\\
\hline
\(5^o\) & +3 &  & &  &  Caracteristica bonus \\
\hline
\(6^o\) & +4 &  & &  & Combate preciso +3 \\
\hline
\(7^o\) & +5 &  & &  & \\
\hline
\(8^o\) & +6 &  & &  & Combate preciso +4 \\
 & &  & &  & Caracteristica bonus\\
\hline
\(9^o\) & +6 &  & &  & \\
\hline
\(10^o\) & +7 &  & &  & Ataque preciso poderoso aprimorado \\
  & & & & & Combate peciso +5 \\
\hline
\(11^o\) & +8 &  & &  & Caracteristica bonus \\
\hline
\(12^o\) & +9 &  & &  & Combate preciso +6\\
\hline
\(13^o\) & +9 &  & &  & \\
\hline
\(14^o\) & +10 &  & &  & Combate preciso +7 \\
 & &  & &  & Caracteristica bonus\\
\hline
\(15^o\) & +11 &  & &  & Combate preciso aprimorado\\
\hline
\(16^o\) & +12 &  & &  & Combate preciso +8 \\
\hline
\(17^o\) & +12 &  & &  & Caracteristica bonus\\
\hline
\(18^o\) & +13 &  & &  & Combate preciso +9\\
\hline
\(19^o\) & +14 &  & &  & \\
\hline
\(20^o\) & +15 &  & &  & Combate preciso +10 \\
 & &  & &  & Caracteristica bonus\\
\hline
\end{tabular}
\end{center}
\caption{O Especialista}
\label{class:espec}
\end{table}
\begin{multicols}{2}


\end{multicols}
