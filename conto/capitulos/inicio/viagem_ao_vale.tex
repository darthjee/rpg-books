\subsection{Viagem ao vale}

Os Lupa Tuhans precisavam atravessar o deserto de Kamir
saindo de seu ber\c{c}o. Lyania junto com Flathan
levaram o grupo em dire\c{c}\~ao \`a Schatten Tal,
atraves do deserto e seus ventos cavalgantes.

No segundo dia de caminhada, ao longe foi avistado um grupo
de cavaleiros ou uma tempestade de areia, uma nuvem
de areia segundo cavaleiros semi-visiveis.
Os caveliros mesclados \`a tempestade avan\c{c}avam
rapida e ca\'oticamente.

O druida for\c{c}ou uma parte do grupo a se esconder sob manto e areia
a fim de fugir da tempestade, enquanto os capazes de combater
ficaram esperando, armas em punho, a chegada dos cavaleiros errantes
que chegaram cavalgando, uivando e atropelando.

O pr\'oprio vento era cavalo, e a areia o cavaleiro,
atropelando esmagando e engolfando os viajantes. Acordando,
os que conseguiam se mover desenterraram os sufucados pela
tempestade.


Logo que o elfo encontrou sua espada perdida, eles retomaram a viagem,
sendo esta marcada pelo encontro com o verme gigante do deserto e
as deciz\~oes do \(4^o\) dia de viagem, quando o estoque de
agua come\c{c}ou a minguar e eles se viram for\c{c}ados mudar a rota
para \textbf{Schatten Palast} onde eles poderiam encontrar agua.

Lyania se mostrou uma profunda conhecedora do deserto
levando o grupo em busca do pal\'acio perdido que so poderia
ser visto sob a tempestade certa para desenterrar seua alvernaria antiga.

\subsection{Pal\'acio de sangue e sal}
Entrar no pal\'acio se revelou simples, a torre de vigia recebeu uma entrada
escondida aberta por c\'odigo secreto que levou os sedentos viajantes areia abaixo
lacrando a entrada e sua \'unica saida.
O pal\'acio tinha cheiro de sal, areia e fantasmas.
Avan\c{c}anddo pelos corredores, varios corpos multilados e conservados
foram encontrados entre os escombros. Corpos de soldados vestindo ricas
armaduras atualmente corroidas pelo tempo, mas conservadas o suficente
para se reconhecer que este ja foi um grande pal\'acio de um grande reino.

Schatten Palast foi esquecido desde de que o ultimo rei foi morto
durante um evento que ja foi removido da memoria dos vivvos.
O sil\^encio foi apenas quebrado pelo som barulho da conversa dos viajantes
que se perguntavam onde poderiam conseguir sua preciosa agua.

Rapidamente eles encontraram um veio, uma goteira e a polvora nescessaria
para abrir esta parede o que levou os viajantes a encontrar a caverna com o pequeno
veio de agua, a vida e acesso \`a saida do pal\'acio.

Reabastecidos, eles seguiram o olfato em dire\c{c}\~ao ao ar fresco e \`a
poss\'ivel saida para encontrar uma antiga caverna cujas estalactites
se encontravam com uas contrapartes no ch\~ao de pedra.

O teto escuro escondia o bater de asas, e as pilastras tornavam
a respira\c{c}\~ao mais ruidosa. As Harpias haviam tomado um novo
protetor, e o devorador cinzento havia encontrado uma nova amea\c{c}a.

A luta consumiu muito suor e sangue, privando o devorador deste
o que deixou a harpias sem rumo durante tempo suficiente para a fuga do
grupo para a saida da caverna.

\subsection{Caravana de areia}
A viagem para o vale prosseguiu, o calor continuava a assolar os viajantes
durante os dias seguintes.
Ao \(3^o\) dia, eles puderam ver as as montanhas se aproximando e ao longe,
e da outra dire\c{c}\~ao, um grupo de viajantes, talvez uma caravana comerciante,
entretanto Lyania advertiu a inexist\^encia de rotas comerciais nesta regi\~ao.

O grupo tentou tomar rota e velocidade a fim de evitar encontro com a
caravana, mas eles precisavam chegar rapidamente nas montanhas
se eles quisessem reabastecer seu estoque d'agua sendo for\c{c}ados assim a manter
a rota que culminaria no encontro dos 2 grupos em 2 dias.

Antes que o grupo tomasse uma nova decis\~ao, eles puderam avistar um grupo de
cavaleiros se aproximando em alta velocidade. Eles haviam sido detectados
e n\~oa conseguiram detectar os batedores a tempo.

Os cavaleiros, 12 deles armados de lan\c{c}as e espadas, 12 de arcos e bestas
cavalgando 12 cavalos com armaduras leves formadas de banboos, canhamo e
o que aparentava ser fios de prata (\ref{enc:ini:cara}).

Os besteiros e arquiros desmontaram a uma dist\^ancia segura para poder
chover suas setas sobre os viajantes, enquanto os cavaleiros avan\c{c}aram
rapidamente.

Durante a luta, Lyania e Flathan cairam presas dos assaltantes
do deserto que levaram aquilo que puderam carregar de volta para a
caravana de mercadores e escravos deixando o grupo com a decis\~ao de
ir diretamente para o vale ou partir em busca dos companheiros perdidos.
