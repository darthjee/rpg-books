\chapter{Regras}
\begin{multicols}{2}
O mundo de \input{constantes/nome_plano.tex} \'e diferente dos v\'arios
planos de exist\^encia, e da mesma forma, as regras que o definem
diferenciam-se dos outros planos

\section{Combate}
O combate apresenta varia\c{c}\~oes que refletem uma proximidade
com a realidade presente.

\subsection{Armaduras e convers\~ao de dano}
As armaduras utilizadas pelos aventureiros em
\input{constantes/nome_plano.tex}
n\~ao apenas impedem que golpes penetrem sua prote\c{c}\~ao,
mas tambem ajudam a amenizar os golpes sofridos
amortecendo o impacto evitando os golpes fatais.

Na pr\'atica, todo golpe f\'isico (dano de cortante,
contundente ou perfurante) que atinge o personagem
ter\'a uma por\c{c}\~ao convertida em dano n\~ao letal.

O calculo para se determinar quanto do dano sera convertido
baseia-se na prote\c{c}\~ao oferecida pela armadura.

O \'indice de convers\~ao de dano \'e
\(\frac{1}{5}\) da prote\c{c}\~ao oferecida pela armadura
(seja ela armadura ou armadura natural).
A prote\c{c}\~ao da armadura n\~ao \'e alterada de forma que esta
convers\~ao \'e um b\^onus.

Caso o personagem tenha um arranjo de armaduras, somam-se os b\^onus
antes de se calcular a convers\~ao de dano.

Toda vez que o personagem \'e atingido, converte-se
uma quantidade de dano letal em dano n\~ao letal
igual ao \'indice de convers\~ao de dano do personagem.

Apenas as armaduras e armaduras naturais, ignorando
seu b\^onus m\'agico, oferecem esta convers\~ao de dano
j\'a que esquiva, deflex\~ao e armaduras feitas de energia
m\'agica n\~ao oferecem um amortecimento.
\end{multicols}
