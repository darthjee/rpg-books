\chapter{Defini\c{c}\~oes}
%\begin{multicols}{2}
Aika Siirtomaa \'e um cen\'ario que pode ser utilizado com
qualquer outro cen\'ario existente.

Aika nasce a partir de um apocalipse mmedieval que alterou toda a
geografia, cosmologia e a organiza\c{c}\~ao da pr\'opria sociedade.

O cap\'itulo de defini\c{c}\~oes foi feito para ajudar o pr\'oprio
autor no entendimento de como o cen\'ario foi criado.

\section{Nomenclaturas}
A nomenclatura de ra\c{c}as, cidades, personagens e outros elementos
do mundo utiliza os idiomas das pr\'oprias ra\c{c}as que se
baseia em idiomas existente nas nossa pr\'opria realidade.

Os idiomas que deram origem aos idiomas falanos no plano
podem ser encontrados na tabela \ref{def:nomen:lang}

\begin{table}[hbt]
\begin{center}
\caption{Idiomas base}
\begin{tabular}{|c|c|}
\hline
\textbf{Ra\c{c}a / Idioma} & \textbf{Idioma base} \\
\hline
Humanos & Italiano \\
	& Latim \\
	& Frances \\
\hline
Tontut & Finlandes \\
\hline
An\~ao & Alem\~ao \\
\hline
Halfling & Gales \\
Hobbit & \\
\hline
Pass Halaal & \'Arabe \\
\hline
\end{tabular}
\label{def:nomen:lang}
\end{center}
\end{table}

\section{C\'alculos e tabelas}
O mundo de Aika tem grande rela\c{c}\~ao com a realidade
na sua constru\c{c}\~ao, e para tal v\'arias regras foram
definidas.

\subsection{Dist\^ancia de visualiza\c{c}\~ao}
\begin{equation}
\begin{array}{c}
f(h) = \sqrt{2Hh+h^2} \\
X(v,o) = f(v)+f(o) \\
\end{array}
\end{equation}

\begin{equation}
\begin{array}{c}
g(H,v,o) =\arccos\left(\frac{(H+v)^2 + (H+o)^2-X(v,o)^2}{2(H+v)(H+o)}\right) \\
\phi = g(H,v,o) \\
\phi \approx \lim_{v<<<H,o<<<H}g(H,v,o) = \frac{X}{H} \\
d = \phi H \approx X
\end{array}
\end{equation}

\begin{equation}
\end{equation}
%\end{multicols}
